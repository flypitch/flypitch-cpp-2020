%% For double-blind review submission, w/o CCS and ACM Reference (max submission space)
\documentclass[sigplan,10pt,review, anonymous]{acmart}

% note(jesse, October 01 2019, 04:00 PM): I'm getting weird typesetting authors when the `anonymous' option is removed
 
% add: anonymous
\settopmatter{printfolios=true,printccs=false,printacmref=false}
%% For double-blind review submission, w/ CCS and ACM Reference
%\documentclass[sigplan,review,anonymous]{acmart}\settopmatter{printfolios=true}
%% For single-blind review submission, w/o CCS and ACM Reference (max submission space)
%\documentclass[sigplan,review]{acmart}\settopmatter{printfolios=true,printccs=false,printacmref=false}
%% For single-blind review submission, w/ CCS and ACM Reference
%\documentclass[sigplan,review]{acmart}\settopmatter{printfolios=true}
%% For final camera-ready submission, w/ required CCS and ACM Reference
%\documentclass[sigplan]{acmart}\settopmatter{}


%% Conference information
%% Supplied to authors by publisher for camera-ready submission;
%% use defaults for review submission.
\acmConference[CPP'20]{The 9th ACM SIGPLAN International Conference on Certified Programs and Proofs}{January 20--21, 2020}{New Orleans, LA, USA}
\acmYear{2020}
\acmISBN{} % \acmISBN{978-x-xxxx-xxxx-x/YY/MM}
\acmDOI{} % \acmDOI{10.1145/nnnnnnn.nnnnnnn}
\startPage{1}

%% Copyright information
%% Supplied to authors (based on authors' rights management selection;
%% see authors.acm.org) by publisher for camera-ready submission;
%% use 'none' for review submission.
\setcopyright{none}
%\setcopyright{acmcopyright}
%\setcopyright{acmlicensed}
%\setcopyright{rightsretained}
%\copyrightyear{2018}           %% If different from \acmYear

%% Bibliography style
\bibliographystyle{ACM-Reference-Format}
%% Citation style
%\citestyle{acmauthoryear}  %% For author/year citations
%\citestyle{acmnumeric}     %% For numeric citations
%\setcitestyle{nosort}      %% With 'acmnumeric', to disable automatic
                            %% sorting of references within a single citation;
                            %% e.g., \cite{Smith99,Carpenter05,Baker12}
                            %% rendered as [14,5,2] rather than [2,5,14].
%\setcitesyle{nocompress}   %% With 'acmnumeric', to disable automatic
                            %% compression of sequential references within a
                            %% single citation;
                            %% e.g., \cite{Baker12,Baker14,Baker16}
                            %% rendered as [2,3,4] rather than [2-4].


%%%%%%%%%%%%%%%%%%%%%%%%%%%%%%%%%%%%%%%%%%%%%%%%%%%%%%%%%%%%%%%%%%%%%%
%% Note: Authors migrating a paper from traditional SIGPLAN
%% proceedings format to PACMPL format must update the
%% '\documentclass' and topmatter commands above; see
%% 'acmart-pacmpl-template.tex'.
%%%%%%%%%%%%%%%%%%%%%%%%%%%%%%%%%%%%%%%%%%%%%%%%%%%%%%%%%%%%%%%%%%%%%%


%% Some recommended packages.
\usepackage{booktabs}   %% For formal tables:
                        %% http://ctan.org/pkg/booktabs
\usepackage{subcaption} %% For complex figures with subfigures/subcaptions
%% http://ctan.org/pkg/subcaption

\usepackage{color}
\definecolor{keywordcolor}{rgb}{0.7, 0.1, 0.1}   % red
\definecolor{tacticcolor}{rgb}{0.1, 0.2, 0.6}    % blue
\definecolor{commentcolor}{rgb}{0.4, 0.4, 0.4}   % grey
\definecolor{symbolcolor}{rgb}{0.0, 0.1, 0.6}    % blue
\definecolor{sortcolor}{rgb}{0.1, 0.5, 0.1}      % green
\definecolor{attributecolor}{rgb}{0.7, 0.1, 0.1} % red

\usepackage{listings}
\def\lstlanguagefiles{lstlean.tex}
\lstset{language=lean,breakatwhitespace,xleftmargin=\parindent}
\usepackage{stmaryrd}

\begin{document}

%% Title information
\title{A Formal Proof of the Independence of the Continuum Hypothesis}
                                        %% [Short Title] is optional;
                                        %% when present, will be used in
                                        %% header instead of Full Title.
% \titlenote{with title note}             %% \titlenote is optional;
                                        %% can be repeated if necessary;
                                        %% contents suppressed with 'anonymous'
% \subtitle{Subtitle}                     %% \subtitle is optional
% \subtitlenote{with subtitle note}       %% \subtitlenote is optional;
                                        %% can be repeated if necessary;
                                        %% contents suppressed with 'anonymous'


%% Author information
%% Contents and number of authors suppressed with 'anonymous'.
%% Each author should be introduced by \author, followed by
%% \authornote (optional), \orcid (optional), \affiliation, and
%% \email.
%% An author may have multiple affiliations and/or emails; repeat the
%% appropriate command.
%% Many elements are not rendered, but should be provided for metadata
%% extraction tools.

%% Author with single affiliation.
\author{Jesse Michael Han}
% \authornote{with author1 note}          %% \authornote is optional;
                                        %% can be repeated if necessary
% \orcid{nnnn-nnnn-nnnn-nnnn}             %% \orcid is optional
\affiliation{
  % \position{Position1}
  \department{Department of Mathematics}              %% \department is recommended
  \institution{University of Pittsburgh}            %% \institution is required
  \streetaddress{4200 Fifth Ave}
  \city{Pittsburgh}
  \state{PA}
  \postcode{15260}
  \country{USA}                    %% \country is recommended
}
\email{jessemichaelhan@gmail.com}          %% \email is recommended

%% Author with two affiliations and emails.
\author{Floris van Doorn}
% \authornote{with author2 note}          %% \authornote is optional;
                                        %% can be repeated if necessary
\orcid{0000-0003-2899-8565}             %% \orcid is optional
\affiliation{
  % \position{Position2a}
  \department{Department of Mathematics}              %% \department is recommended
  \institution{University of Pittsburgh}            %% \institution is required
  \streetaddress{4200 Fifth Ave}
  \city{Pittsburgh}
  \state{PA}
  \postcode{15260}
  \country{USA}                    %% \country is recommended
}
\email{fpvdoorn@gmail.com}         %% \email is recommended

%% Abstract
%% Note: \begin{abstract}...\end{abstract} environment must come
%% before \maketitle command
\begin{abstract}
Text of abstract \ldots. %TODO
\end{abstract}


%TODO
%% 2012 ACM Computing Classification System (CSS) concepts
%% Generate at 'http://dl.acm.org/ccs/ccs.cfm'.
% \begin{CCSXML}
% <ccs2012>
% <concept>
% <concept_id>10011007.10011006.10011008</concept_id>
% <concept_desc>Software and its engineering~General programming languages</concept_desc>
% <concept_significance>500</concept_significance>
% </concept>
% <concept>
% <concept_id>10003456.10003457.10003521.10003525</concept_id>
% <concept_desc>Social and professional topics~History of programming languages</concept_desc>
% <concept_significance>300</concept_significance>
% </concept>
% </ccs2012>
% \end{CCSXML}

% \ccsdesc[500]{Software and its engineering~General programming languages}
% \ccsdesc[300]{Social and professional topics~History of programming languages}
%% End of generated code


%TODO
%% Keywords
%% comma separated list
\keywords{forcing, formal proof, independence, continuum hypothesis}  %% \keywords are mandatory in final camera-ready submission


%% \maketitle
%% Note: \maketitle command must come after title commands, author
%% commands, abstract environment, Computing Classification System
%% environment and commands, and keywords command.
\maketitle


\section{Introduction}
\label{section:intro}

Introduce CH, forcing, boolean semantics, Lean~\cite{de2015lean}. Start can be similar to ITP paper.

Building on the work of Han and van Doorn \cite{DBLP:conf/itp/HanD19}, we complete a formal proof of the independence of the continuum hypothesis.
%(for the purposes of lightweight double-blind reviewing we should not call it ``our'' paper, even though that is obvious)

\subsection{Proof Outline}
\label{subsection:intro:outline}
Boolean-valued semantics, Forcing, cardinal collapse, independence of CH

\section{First-order logic}
\label{section:fol}

\subsection{Terms, Formulae and Proofs}
\label{subsection:fol:terms}

Our encoding of terms/formulas/proofs in FOL, similar to ITP paper.
Maybe ordinary semantics and soundness/completeness.

\subsection{ZFC}
\label{subsection:fol:zfc}

set theory as a first-order language and ZFC as a first-order theory. Statement of CH.

\section{Boolean-valued semantics}
\label{section:boolean-semantics}

boolean-valued semantics, boolean-valued soundness,

\subsection{Boolean valued models of ZFC}
\label{subsection:fol:bset}

boolean-valued model of ZFC

\section{Collapse forcing}
% note(jesse, October 01 2019, 04:11 PM): i think levy collapse only involves the collapse of strongly inaccessible cardinals, and that the collapsing argument we use goes by the name of σ-closed, ω-closed, countably closed forcing or maybe just collapse forcing? we should clarify this.
\label{section:collapse}

partial functions in Lean, introduce the collapsing boolean algebra, show omega-closedness.

\section{Forcing} \label{section:forcing}
\subsection{Boolean-valued models of set theory in dependent type theory}

Our starting point is the Aczel encoding of \(\mathsf{ZFC}\) (\cite{aczel1978type, aczel1986type, aczel1982type}) into dependent type theory, later implemented in Coq by Werner \cite{werner1997sets}, and in Lean's \lstinline{mathlib} by Carneiro \cite{mario1}. The idea is to take a type universe \lstinline{Type u} and imitate the cumulative hierarchy construction by an inductive type:
\begin{lstlisting}
inductive pSet : Type (u+1)
| mk (α : Type u) (A : α → pSet) : pSet
\end{lstlisting}
Note that \lstinline{mk empty empty.elim} always exists, and plays the same role as \(\emptyset\) at the bottom of the cumulative hierarchy.


% Whatever is important in \lstinline{bvm}, \lstinline{bvm_extras}, \lstinline{bvm_extras2} for this argument.

\subsection{Construction of $\aleph_1$} \label{subsection:forcing:aleph-1}

Construction of $\aleph_1$.

\subsection{The independence of CH} \label{subsection:forcing:independence}

The final argument to show that this particular model satisfies CH.

\section{Related Work}
\label{section:related-work}
Compare with other formalization on the consistency of CH.


\section{Conclusions}
\label{section:conclusions}
Future work: construction of L, ZFC without function symbols, parser and better printer for FOL

%% Acknowledgments
\begin{acks}                            %% acks environment is optional
                                        %% contents suppressed with 'anonymous'
  %% Commands \grantsponsor{<sponsorID>}{<name>}{<url>} and
  %% \grantnum[<url>]{<sponsorID>}{<number>} should be used to
  %% acknowledge financial support and will be used by metadata
  %% extraction tools.
  The authors gratefully acknowlege the support by the
  \grantsponsor{GS100000001}{Alfred P. Sloan Foundation}{https://doi.org/10.1038/201765d0}, Grant
  No.~\grantnum{GS100000001}{G-2018-10067}.
\end{acks}

%% Bibliography
\bibliography{flypitch-cpp}

%% Appendix
\appendix
\section{Appendix}

Text of appendix \ldots

\end{document}
